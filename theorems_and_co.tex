\usepackage{tcolorbox}
\tcbuselibrary{skins}
\tcbuselibrary{theorems}

% Theorem
\newtcbtheorem[number within=section]{theorem}{Theorem}%  
{colframe=blue!50!black,colback=yellow!15!white!95!brown,coltitle=red!50!black,
drop fuzzy shadow=blue!50!black!50!white,boxrule=0.4pt,enhanced,
fonttitle=\upshape\bfseries,fontupper=\upshape,
theorem style=break,terminator sign none}{theo}

% Proposition
\newtcbtheorem[use counter from=theorem]{proposition}{Proposition}%
{colframe=blue!50!black,colback=yellow!20!white,coltitle=red!50!black,
drop fuzzy shadow=blue!50!black!50!white,boxrule=0.4pt,enhanced,
fonttitle=\upshape\bfseries,fontupper=\upshape,
theorem style=break,terminator sign none}{pro}

% Corollary
\newtcbtheorem[use counter from=theorem]{corollary}{Corollary}%
{colframe=green!60!black,colback=green!60!black!10!white,coltitle=green!60!black,
boxrule=0.4pt,
fonttitle=\upshape\bfseries,fontupper=\upshape,
theorem style=break,terminator sign none}{coro}

% Korollar
\newtcbtheorem[use counter from=theorem]{korollar}{Korollar}%
{colframe=green!60!black,colback=green!60!black!10!white,coltitle=green!50!black,
boxrule=0.4pt,
fonttitle=\upshape\bfseries,fontupper=\upshape,
theorem style=break,terminator sign none}{koro}

% Lemma
\newtcbtheorem[use counter from=theorem]{lemma}{Lemma}%
{colframe=green!60!black,colback=green!60!black!10!white,coltitle=green!50!black,
boxrule=0.4pt,
fonttitle=\upshape\bfseries,fontupper=\upshape,
theorem style=break,terminator sign none}{lem}

% Definition
\newtcbtheorem[number within=section]{definition}{Definition}%
{colframe=blue!50!black,colback=cyan!10!white,coltitle=blue!50!black,
boxrule=0.4pt,
fonttitle=\upshape\bfseries,fontupper=\upshape,
theorem style=break, terminator sign none}{def}

% Claim
\newtcbtheorem{claim}{\textbf{Claim ---}}%
{coltitle=green!50!black,colback=green!60!black,boxrule=0pt,
frame hidden, sharp corners, enhanced,
borderline west={3pt}{0pt}{green!60!black},opacityback=0.1,
theorem style=plain,terminator sign none}{cl}

% Behauptung
\newtcbtheorem{behauptung}{\textbf{Behauptung ---}}%
{coltitle=green!50!black,colback=green!60!black,boxrule=0pt,
frame hidden, sharp corners, enhanced,
borderline west={3pt}{0pt}{green!60!black},opacityback=0.1,
theorem style=plain,terminator sign none}{Be}

% Remark
\newtcbtheorem{remark}{\textbf{Remark ---}}%
{coltitle=black,colback=white!60!black,boxrule=0pt,
frame hidden,sharp corners, enhanced,
borderline west={3pt}{0pt}{black},
opacityback=0.1,
theorem style=plain,terminator sign none}{Re}

% Example
\newtcbtheorem[number within=section]{example}{Example}%
{colframe=red!50!white,colback=red!5!white,coltitle=red!50!black,boxrule=0.4pt,
enhanced, sharp corners,
fonttitle=\upshape\bfseries,fontupper=\upshape,
theorem style=break,terminator sign none}{ex}

% Beispiel
\newtcbtheorem[number within=section]{beispiel}{Beispiel}%
{colframe=red!50!white,colback=red!5!white,coltitle=red!50!black,boxrule=0.4pt,
enhanced, sharp corners,
fonttitle=\upshape\bfseries,fontupper=\upshape,
theorem style=break,terminator sign none}{bei}



\newcommand{\N}{\mathbb{N}}
\newcommand{\Z}{\mathbb{Z}}
\newcommand{\Q}{\mathbb{Q}}
\newcommand{\R}{\mathbb{R}}
\newcommand{\C}{\mathbb{C}}
